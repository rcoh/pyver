\documentclass{article}[12pt]
\usepackage{fullpage}
\usepackage{setspace}
\usepackage{minted}
\author{Russell Cohen}
\title{Pyver: \\ A Python Static Analyzer Written in Haskell}
\date{\today}
\begin{document}
\renewcommand{\theFancyVerbLine}{
      \sffamily\textcolor[rgb]{0.5,0.5,0.5}{\scriptsize\arabic{FancyVerbLine}}}

\maketitle
\section{Motivation}

\subsection{Compiled vs. Interpreted Languages}
\doublespacing

Python is an extremely widely used programming language; it's utilized everywhere -- from
introductory computer science courses to production web sites to scientific computing. Its strength
lies in its flexibility, readability and simplicity. In order to facilitate these features, Python
is a dynamic language that is interpreted at runtime. While this enables much of its
flexibility, it also invites a class of potential programming errors to sneak into one's code that
would be quickly detected in compiled languages.

Two classes of errors can occur while writing software: logical errors in a semantically correct program and
errors that violate the semantics of the programming language. The study of detecting logical errors has been
increasing with the development of self-proving languages such as Coq and Agda, but remains a somewhat open problem.
Semantic errors, on the other hand, are well understood. Compilers for static languages can reason about these
errors at compile time and won't produce a runnable binary until the errors are resolved. Python, since it isn't
compiled, lacks these protections.

In general, the programmer wants to discover errors as quickly as possible. The sooner the
programmer finds the error, the sooner the programmer can investigate the problem, attempt to fix it, and
continue their development process. Compiled languages detect semantic errors during the compile phase
-- before the program is even run. This increases the time before the program can be run, but it
\emph{guarantees} that the program is free from semantic errors. Languages like Java, C++, and Haskell
fall into this category. Once a Java program is compiled, the compiled binary is guaranteed to be
free from type errors, provided that the code avoids certain unsafe operations.
Figure~\ref{fig:csemanticerror} gives an example of a semantic error that would be caught by the
compiler in the \texttt{C} programming language. 

\begin{figure} \begin{minted}{c} 
    struct Student { 
      int grade; 
    } 
    Student s; 
    int atten = s.attendance; // Semantic error -- Student has no field attendance 
  \end{minted} 
  \caption{A semantic error in the C Programming Language. This is
    detected by the compiler at compile time.} 
  \label{fig:csemanticerror}
\end{figure}

\begin{figure}
    \begin{minted}{python}
        class Student:
          attendance = 100

        s = Student()
        atten = s.attendance # Runtime type error
    \end{minted}
    \caption{A semantic error in Python isn't caught until runtime which could an occur and
    arbitrary amount of time after starting the program.}
    \label{fig:pythonsemanticerror}
\end{figure}

Runtime-interpreted languages, however, lack these guarantees. Rather than being compiled,
interpreted languages are evaluated line by line as the interpreter reaches the line in question.
Interpreted languages can be extremely flexible and allow programmers large amounts of freedom in
how they use language constructs. However, these languages make weak guarantees on the programs they
run. In Python, for example, the interpreter only checks that the program is syntactically valid
before running it. Figure~\ref{fig:pythonsemanticerror} gives an example of a semantic error in
Python which will not be caught until runtime. These present an entire class of errors that can only
be discovered when the interpreter attempts to execute the offending statement. In short-running
programs, the debug time can be equivalent to a compiled language; the programmer runs her code and
the Python interpreter immediately outputs a type error. If instead, the errant statement doesn't
get executed until several hours after the program began, the programmer has now wasted hours that
could have been saved, had she been notified of a type error before running the program. 

Late error detection also poses problems when using Python in a web application or other target with rarely executed
code paths -- if a type error is accidentally introduced in a seldom used code path, the error won't be discovered until
much later when an unsuspecting user hits the code path and is presented with an error. Ideally, the programmer could have been
alerted of this error before deploying the code. If the website had been written in a statically typed language, the
error would have been caught at compile time and fixed. Instead, it managed to make it all the way to an end user.

\section{Why Doesn't Python Include Semantic Checking?}
In light of the problems with dynamic languages such as Python, one could wonder why Python is missing
static checking of types. The problem lies in the flexibility of Python's type system. 

Before investigating the flexibility of the type system, let's clarify exactly what we mean when we
discuss the type of a variable. Intuitively, it's any property where changing it could cause a type
error. In Python, this encompasses the fields an object has and its underlying core type.
Figure~\ref{fig:pythontypes} details several different types in Python.

\begin{figure}
    \begin{minted}{python}
        a = [] # A has list type
        b = 5 # B has int type
        c = a + b # This is a runtime type error

        # class X has one integer field, a
        class X:
          a = 5

        x_inst = X() # x_inst has one integer field a

        # This would be a runtime type error:
        # val = x_inst.b
        
        # Add another field to the instance
        x_inst.b = True # x_inst is now A DIFFERENT TYPE. 
        
        # Now, not a type error:
        val = x_inst.b
    \end{minted}
    \caption{Examples of types in Python. Types can either be a core type or instance type. Instance
    types vary in the fields they define}
    \label{fig:pythontypes}
\end{figure}

In Python, variables are dynamically typed. This means that the type of a variable can change as
the program proceeds. Contrast this with Java and other statically typed languages where a variable
is born with a type that never changes. Since Python types can change, it stands to reason that if
we wanted to, we could encode information in the type of a variable rather than using the value. You
could write an entire program that worked by only modifying the types of variables!  Although I will
not prove it formally, it should be relatively obvious given this information that the only perfect
type verification system for Python is to run the program.  After you've run the program, you have
defeated the purpose of your verification system (since running and waiting for the program was what we
were trying to avoid in the first place).

Though a perfect bounded-time simulation of the type system is impossible, it is possible to create
a system that approximates type information instead. Ideally, rather than running in an undecidable
amount of time, our type analyzer will run in time linear in the number of statements in the
program. From this, we can see that we must not process statements a non-constant number of times.
For example, if our system directly simulated a while loop, that would violate this guarantee, since
there is no bound on how many iterations the loop can have.

Thus, we must approximate control flow that would cause our program to loop, as a loop removes the
bound of constant work per statement. There are two primary elements of control flow that would
cause our program to loop on itself: looped control flow and recursion. Handling looped control is
actually quite straightforward: simply ignore it -- execute the loop body as if the 
control structures didn't exist, and start the looping variables at their initial type.
Figure~\ref{fig:ignoreloops} gives an example where no information is lost by executing the loop
body only once. Unfortunately, it's very easy to create examples where this isn't correct --
Figure~\ref{fig:cantignoreloops} shows an example where our inference cannot produce the correct
result.

\begin{figure}
    \begin{minted}{python}
      def count_occurrences(list, item_to_count):
          count = 0
          for item in list_of_data:
            if item == item_to_count: 
              count += 1
          return count
      
      some_strings = ['6.01', '6.02', '6.035', '6.UAP', '6.UAT', '6.UAP']
      count = count_occurrences(some_strings)

  \end{minted}
  \caption{A loop to count the occurrences of an item in a list. In this loop, the types of the
      variables can be fully inferred by one evaluation of the loop.
  }
  \label{fig:ignoreloops}
\end{figure}


\begin{figure}
    \begin{minted}{python}
        class Empty:
            pass

        e = Empty()
        for i in range(10):
            setattr(e, "prop" + str(i), True) # e.prop1 = True, e.prop2 = True, etc.

        print e.prop5
        # The type of e now is directly dependent on the number of loop iterations
    \end{minted}
    \caption{Performing a silly computation with types. Without fully evaluating the loop, there is no
        way that we could determine the type of \texttt{e}. If we evaluated once, we would infer
        \texttt{e} to have only one parameter, \texttt{prop1}. However, \texttt{e} will have 10
        properties set.  
     } 
     \label{fig:cantignoreloops}
 \end{figure}

When our analyzer reaches a loop, it registers the type information it can infer about the looping
variables and it evaluates the body of loop a single time. Then, it makes the simplifying (and
potentially incorrect) assumption that though the values may change in subsequent executions of the
loop, the type information will not. At this point, we could also use other approximation
techniques, such as using a bounded-iteration fixed point algorithm. We would iterate over the loop
until the types stopped changing, being sure to stop at predetermined time if the fixed point
did not appear. However, loops creating different types per iteration in real code is quite rare, so we
don't handle it here.

The more complex case to handle is that of recursion -- recall that iteration can even be simulated
by recursive functions. We handle recursive functions by first detecting that a function is
recursive by tracking the function call graph, and then applying an approximation. We approximate
recursive functions by assuming (possibly incorrectly) that the concrete type returned by the
function is the actual return type of the function in \emph{all} branches. From their, we're able
to escape the recursion, since we can replace the recursive call with the type of the base-case.
Figure~\ref{fig:factorial} gives an example of a function where this does not lose information.

\begin{figure}
    \begin{minted}{python}
        def factorial(n):
          if n == 1:
            return n
          else:
            return n * factorial(n - 1)

        factorial(5)
    \end{minted}
    \caption{We can't simply keep following the control flow to find the type of \texttt{factorial} -- the
        recursion would loop forever. Rather, when we see the recursive call, we make the assumption
    that it returns that same type as the other return statement, in this case an integer.}
    \label{fig:factorial}
 \end{figure}

\section{High Level Design} Our static analyzer works by creating an abstract representation of
Python programs and simulating their execution within this abstract space. Our abstract
representation comprises of two components: the type system, and the action system. The type
system fully represents our view into the state of a Python program. To create the
type representation for a given program, we translate each line of code into a series of composable
actions.  These actions are then applied to the program state, yielding a new state. At each step in
the state, we evaluate the state for possible problems. The system is
designed to find two types of problems in code: type errors which would cause the program to halt in
an unexpected way, and accidental types that will not halt the program but will almost certainly
cause unexpected behavior to occur. The detection of accidental but non-fatal errors is a currently
investigated field; the Clang C/C++ compiler presents a productionized example of this practice and
is under currently active development.\footnote{http://clang.llvm.org/diagnostics.html}

\subsection{Handling of Errors}

Error handling must be done with care to create a verification system that is useful in practice.
For example, if the analyzer stops running upon reaching a potential error, it's of limited use,
especially if the error was a false positive. In our system, errors are actually included in the
resulting type information. Once the analysis is complete, we can collect all variables whose type
is inferenced to be in error and present them coherently to the user. For example, consider the
following short Python program:

\begin{minted}{python}
    x = a
\end{minted}
In this program, \verb=a= is assigned to \verb=x=, but \verb=a= isn't defined. The type of
\verb=x= would then become \verb=NotDefined 'a'=. Any further computations that used \verb=x= would
have this value propagated.

This behavior is identical for other classic type mistakes, eg.:
\begin{minted}{python}
    a = 5
    x = a[1] # IdentifierNotSubscriptable 'a'
\end{minted}
In this example, we assign \verb=x= the type \texttt{IdentifierNotSubscriptable} indicating that the attempt to
list-index against \texttt{a} was invalid.

Along with being propagated through the program state, we also maintain a list of errors that have
been detected as the program executes. This is done for convenience and to prevent the potential
snowball effect from creating an overwhelming number of errors presented to the user stemming from a
single typo.

\subsection{Handling Of Warnings}

Since Pyver does a reasonable job a producing a full picture of the types used in a program, it can
also detect usages of types that are probably unintentional. We leverage this knowledge to warn
programmers about suspect code which can help limit time spent debugging. To produce warnings, we
have added two heuristics. In a complete system, many more would help to catch bugs in a wider range
of programs. Pyver can currently produce the following warnings:

\begin{itemize}
    \item \textbf{Differing core types surrounding \texttt{==}}: In Python, an integer type will never
          equal a list type will never equal a set type will never equal a dictionary type will
          never equal a string type. This only applies to the core types, however, since their
          equality method cannot be overridden in user defined types. This has one notable edge case: \texttt{True} and
          \texttt{False} in
          Python are not really their own types -- \texttt{True} is an alias for 1 and
          \texttt{False} is an alias for 0. Therefore, comparing integers to booleans is a
          reasonable thing to do (although a bit opaque to the reader).

      \item \textbf{Using a value from a function that returns only \texttt{NoneType}}: While functions can often
          return \texttt{NoneType} validly in some cases, our system will detect a function that can only
          return None in cases where there are no return statements. Using the value in a function
          without a return statement is almost certainly incorrect.
\end{itemize}

Python contains many more of these type ``gotchas'', and they could be incorporated into a
productionized version of this system.

\subsection{Representation of Type System} 

Loosely speaking, the type system represents the noun of our inference system. The type system has
the following representation:
\emph{This listing currently represents types in the Haskell style which may be a bit unfamiliar to
some readers. The type name comes first, followed by the types of its fields.}
\begin{minted}{haskell}
  PyType = 
      -- Core Types
      PyInt | 
      PyBool | 
      PyChar |
      PyStr  |
      NoneType |
      FunctionType { name :: String, args :: [PyType], body :: [Action] } | 

      -- Core datastructures
      PyList PyType | 
      PySet  PyType |
      PyDict PyType PyType | 


      -- Normal classes / objects
      ComplexType { String => PyType } | 

      UnionType [PyType] | 
      Identity |
      Unknown |

      AttributeNotFound String |
      IdentifierNotFound String |
      ObjectNotSubscriptable String |
      ObjectNotCallable String 
\end{minted}

This encompasses types for the core Python primitives (ints, bools, strings), the standard
parameterized types (lists, sets, and dictionaries). The \texttt{UnionType} allows us to represent expressions that may be several different types at a given program point, for example, in code like the following:

\begin{minted}{python}
    a = [1, False, []]
    b = a[math.randint(2)]
\end{minted}

Our system infers \verb=b= to be a \verb=UnionType [PyInt, PyBool, PyList Unknown]=.

Union types can also be created when types differ at a branch:
\begin{minted}{python}
    class Student:
      grade = 100

    s_attend_class = math.randbool()
    s = Student()
    if s_attended_class:
      s.attendance = 100
\end{minted}
At the end of this code path, our system infers that \texttt{s} is the union of an object with an attendance field and an
object with just a student field. Further operations will be tried against both representations. Full details on the
semantics of union types are discussed later.

\verb=ComplexType=, which contains a map from \verb=String= to \verb=PyType= is used to represent
classes composed of the other core types. For example:
\begin{minted}{python}
    class Student:
      grade = 95 
      enrolled = True
\end{minted}

Our system infers \verb=Student= to be a 
\verb=ComplexType { grade -> PyInt, enrolled -> PyBool }=

The identity type is used to represent fields which have the identical type of their parent. We use
this in the constructor field of classes. For example, returning the student example, the actual
representation would be
\verb=ComplexType { grade -> PyInt, enrolled -> PyBool, __init__ -> Identity }=
(\verb=__init__= is the constructor method in Python).

Unknown is used when the type of a variable cannot be knowable at a code position, for example:
\begin{minted}{python}
    a = []
\end{minted}
Our system infers \verb=a= to be a \verb=ListType Unknown=.

The remaining types are generated when the action that is applied to a type is illegal to apply (for
example applying \verb=getSubscript= to a \verb=PyInt=.

\subsection{Representation of the action system}
The action system represents the verbs of our abstract representation. The actions correspond to
actions that are built into the Python language. Python programs are
deconstructed into the following actions (summarized below):
\begin{verbatim}
  Action =
    Assignment   lhsFunc :: (PyType -> PyType -> PyType), rhs :: PyType
    GetAttribute item :: String, base_object :: PyType 
    SetAttribute item :: String, attr_value :: PyType, base_object :: PyType 
    DeleteAttribute item :: String, base_object :: PyType
    GetSubscript index ::  PyType,  base_object :: PyType 
    SetSubscript index ::  PyType, value :: PyType, base_object :: PyType 
    DeleteSubscript index ::  PyType, base_object :: PyType 

    ArithPlus left :: PyType, right :: PyType 
    ArithMinus left :: PyType, right :: PyType 
    ArithTimes left :: PyType, right :: PyType
    ArithDiv left :: PyType, right :: PyType 

    ArithPlusEqual left :: PyType, right :: PyType
    ArithMinusEqual left :: PyType, right :: PyType

    ArithCheckEqual left :: PyType, right :: PyType

    FunctionCall callable :: PyType, args :: [PyType]

    LoopFor  looping_scope :: PyType, body :: [Action]
    LoopWhile  looping_scope :: PyType, body :: [Action]
\end{verbatim}

These operations are generalized in that they operate identically not only on concrete types, but
also on scopes. All actions are ``Applyable''. When applied, all actions produce a new PyType. The apply, combined with
the lazyness of Haskell allows us to nest actions -- one action can depend on the result of another
without having to apply it immediately. When the
concrete type is desired, we pass in a scope and apply it, which
produces a new PyType.

Below, we will summarize the Python code that leads to these actions.

\verb=GetAttribute= is the ``dot'' operator as well as simply looking up a variable in a scope. When
applied, GetAttribute yields the type of the resulting attribute.
Accessing in a scope:
\begin{minted}{python}
    a = 5
    x = a # GetAttribute ``a'' GlobalScope
\end{minted}

Accessing in a class:
\begin{minted}{python}
    class Student:
      grade = 95

    s = Student()
    grade = s.grade #GetAttribute ``grade'' s
\end{minted}

Similarly, \verb=SetAttribute= sets class attributes as well as variables into scopes. When applied,
SetAttribute returns the \verb=base_object= with the new attribute added.

\begin{minted}{python}
    a = 5 #SetAttribute {item: 'a',  attr_value: PyInt, base_object : ProgramScope}
    class Student:
      grade = 100

    s = Student()
    s.attendance = .99 # SetAttribute {item: 'attendance':, attr_value: PyFloat, base_object: s}
\end{minted}

\verb=GetSubscript= and \verb=SetSubscript= are the parallel of \verb=GetAttribute= and
\verb=SetAttribute=, but for the subscript operator (\verb=[]=) instead of the dot operator
(\verb=.=).

\begin{minted}{python}
    a = [1, 2, 3]
    b = a[1] # GetSubscript { index :: PyInt, base_object :: a }
    a[2] = b # SetSubscript { index :: PyInt, value : : PyInt, base_object :: PyList Int }
\end{minted}

\section{Implementation Details}

Under the hood, the analyzer is implemented in Haskell. Haskell was chosen because of its type
safety, functional properties, and the presence of a Python parser in the standard library. 

The entire analysis is implemented as one large fold. The program is processed statement by statement. Each
statement produces a new program state, which is given to the next statement.

\subsection{Handling Mutable State}

Representing a language like Python that relies heavily on shared mutable state is tricky in
Haskell. We handle it by using value numbering. If you assign a mutable variable to another in
Python, not only do they share values, but they actually share types. Their types will be identical
for the rest of the time that they reference the same object. Because of that, we must link their
types together. 

To handle this in a language without mutable state, we actually represent complex types by
representing each unbound ComplexType we discover in analysis with a new value number. All values
that should be kept identical share the same value-number. Inside the program representation, rather
than store the values, we store their value numbers. When something is modified, it is modified in
the map from value number to \verb=PyType=.  That way, all attributes which reference the same
object will stay in sync when one is updated.  We see an example where this is required for
correctness in Figure~\ref{fig:mutablestate}.

\begin{figure}
\begin{minted}{python}
    class Student:
      grade = 95
    student1 = Student()
    student2 = student1
    student2.attendance = 99
    # student1 should now also have an attendance attribute
\end{minted}
\caption{Reference equality of two objects links their types}
\label{fig:mutablestate}
\end{figure}

\subsection{Handling User Supplied ``Magic Methods''}
One of Python's most powerful features is the ability for users to override so called ``magic methods''. These methods,
started and followed by double underscores, allow the programmer to override seemingly built-in language features for
their own datatypes. For example, if you define an \texttt{\_\_getitem\_\_} on your class, your class will support list
indexing.
\begin{figure}
    \begin{minted}{python}
        class Funky(object):
            def __getitem__(self, index):
                return index

        f = Funky()
        print f["Hello world"] # prints "Hello World"
    \end{minted}
    \caption{Python allows you to override the behavior of different operators for your own classes}
    \label{fig:magicmethods}
\end{figure}

Pyver supports magic methods by attempting to search for an existing use-supplied definition of the appropriate magic
method matching the language construct. For example, in Figure~\ref{fig:magicmethods}, \texttt{f} is a \texttt{ComplexType}, so upon seeing a
\texttt{GetItem} action, it searches for an existing \texttt{\_\_getitem\_\_} to invoke.
  
Each action that Pyver supports maps one-to-one with a ``magic method''. Currently Pyver does not support all magic
methods (only the most common) but could be easily improved to support the full set.

\subsection{Representation of Union Types}

Pyver's representation of union types is a heuristic designed to give more accurate results in
situations where the possible types for a variable diverge. Union types handle situations where the
actual type of a variable can be any number of discrete different types. We present our
representation -- more accurate representations almost certainly exist, but this seems to be
effective for typical Python code. At their core, union types are simply a list of other types. A
variable typed with a union type may be any one of the concrete types in the set. The rules of union types in
Pyver are as follows:

\begin{enumerate}
  \item A union type of a single type is collapsed into a concrete type
  \item When adding a new type to a union type: 
      \begin{itemize}
          \item Unions have set semantics: if a type is identical to an existing type in the union,
              discard it -- it does
              not enter the union twice.
          \item If the union has become too large, this means it must contain many different
              ComplexTypes or types parameterized by ComplexTypes. This is because there are a
              finite number of other types. In this case, to widen the representation, we pick types
              in the union that differ only in a ComplexType component and merge them. We attempt to
              find types that have no overlap (preventing the need to chose if a parameter is of one
              type or anothe). Widening the representation loses information, specifically, by
              hiding future type errors.
      \end{itemize}
  \item A union cannot contain Unknown -- Unknown indicates no knowledge of the type, as is the case in an empty
              list. Since a union must contain 2 or more types, it can never contain Unknown.
  \item A union cannot contain other union types. They should be merged into the root union.
  \item When actions are applied to a union type, they are mapped over all of the child types in the union. This
      produces a new union with the results, some of which may be error types. The rules for merging are
      applied to the new result before producing a new type.
\end{enumerate}

In Fig~\ref{fig:unionexamples} we present several flows of Union types in programs.
\begin{figure}
    \begin{minted}{python}
        a = [] # PyList Unknown
        a.append(5) # PyList Union [Unknown, Int] collapses to PyList Int
        a.append(``3'') # PyList Union [Int String]
        a.append(1) # PyList Union [Int String] -- identical types exist once

        class Student:
          grade = 5
          age = 10

        class Road:
          grade = 3
          age = 1
        
        b = []
        b.append(Student()) # PyList ComplexType { grade :: Int, age ::  Int }
        # Complex types are structural, so identical fields are identical types
        # The assignment below doesn't change the type
        b.append(Road()) # PyList ComplexType { grade :: Int, age ::  Int }
    \end{minted}
    \caption{Examples of Union Type Analysis}
    \label{fig:unionexamples}
\end{figure}


\section{Analysis Results}

Here, we will present several buggy programs and the results of their analysis.

One of the most powerful features of the analysis system is its ability to detect mistakes that are
not actually type errors. For example, in the Figure~\ref{fig:guessnum}, the analysis system infers that the
user equality checks two fundamentally different types. Since this will always produce false, the
system warns that this could be an error.

\begin{figure}
    \begin{minted}{python}
        my_number = 75 # Inferred PyInt
        while 1:
            guess = raw_input("What's your guess?") #Inferred String
            # WARNING: ArithCheckEqual with type String and Int will never be equal 
            if guess == my_number:
                print "Correct"
                break
            elif guess < my_number:
                print "Go higher"
            else:
                print "Go lower"
    \end{minted}
    \caption{Guess my number game. This code will produce a warning because the analyzer correctly
        detects the lack of a conversion of \texttt{raw\_input} \texttt{string} to \texttt{int}.}
    \label{fig:guessnum}
\end{figure}

The system can also detect classic Python type errors such as trying to append an integer to a list
with \texttt{+=} instead of \texttt{append}. We see this in Figure~\ref{fig:findprimes}.
\begin{figure}
    \begin{minted}{python}
        # Program to collect collect primes
        def is_prime(x):
          for i in range(2, x - 1):
            if x % i == 0:
              return False
          return True
        
        primes = []
        look_until = 10000
        for i in xrange(look_until):
          if is_prime(i):
            # i is inferred Int
            # primes is inferred List [Unknown]

            # ArithPlusEqualsInvalidError: Cannot append int to list. 
            # Only iterable objects can be appended to lists.
            primes += i
    \end{minted}
    \caption{A buggy program to find prime numbers}
    \label{fig:findprimes}
\end{figure}

The system can also detect missing return statements in functions, as we see in
Figure~\ref{fig:noreturn}.
\begin{figure}
    \begin{minted}{python}
        def filter_odd(inp):
          ret = []
          for i in inp:
            if i % 2 == 1:
              ret.append(i)
          # Does not return ret

        inp_list = [1, 5, 123, 1231, 32, 63, 412, 2461512, 2354656]

        # odds is inferred to be NoneType
        # WARNING: Function produces NoneType and is used in assignment. Missing return?
        odds = filter_odd(inp_list)
        # ERROR: NoneType can not be iterated over
        for num in odds:
          print num
    
    \end{minted}
    \caption{A silly program to filter odd numbers from a list}
    \label{fig:noreturn}
\end{figure}
\section{Further Work}

In its current state, Pyver has several avenues for improvement: 
\begin{itemize}
      \item Support for more Python actions / ``magic methods''. Currently, when Pyver encounters a
          piece of code it cannot produce an action for, it simply ignores it.
      \item Creation of more warnings based on common detectable programming mistakes
      \item Support for components of real programs such as imports and type information of external
          libraries
\end{itemize}

\section{Conclusion}

Pyver presents an abstract representation and evaluation strategy for statically evaluating the
types in a dynamic language. Rather than simply attempting to assign a type to each variable, it
simulates the execution of the program to deduce type information. It makes critical approximations
to allow the excecution to skip through loops and recursion. These two types of program flow rarely
change the type information in ``normally written'' programs. Furthermore, skipping through these
types of statements by ignoring the loop component allows us to simulate the programs in bounded 
time, rather than in an undecidable amount of time.

With the type information we can infer as we progress through the program, we can predict both true
type errors that would lead to crashes and probable type errors that lead to unexpected behavior for
the programmer.

\end{document}
